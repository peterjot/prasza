% 1. Należy w programie ustawić zgodne kodowanie, w tym przypadku jest UTF-8. 
% 2. Kompilować ten plik za pomocą pdfLatex.


%____________________________________________________________________________________________
%____________________________________________________________________________________________
% Ustawienia pracy.
%____________________________________________________________________________________________

% Obowiązkowo! %%%%%%%%%%%%%%%%%%%%%%%%%%%%%%%%%%%%%%%%%%%

% Definiowanie stylu i formatu pracy.
\documentclass[a4paper,twoside,12pt]{book}	% twoside - dwustronnie		 
\usepackage{polski}		 		% ustawienie języka
\usepackage[T1]{fontenc}
\usepackage{amsfonts, amsmath, amsthm, amssymb}
\usepackage[utf8]{inputenc}	
\usepackage{latexsym}
\usepackage{indentfirst} 				% wcięcia akapitu
\linespread{1.5}					% odstęp między liniami
\usepackage{geometry}
\geometry{lmargin=3.5cm, rmargin=3cm}	% ustawienie marginesów, lewy większy na oprawę

\usepackage{url}
\usepackage{enumerate}				% numeracja
\usepackage{amsopn}
\usepackage{graphicx} 				% grafika
\usepackage{epstopdf}				% grafika formatu eps
\usepackage{tabularx} 				% tabele
\usepackage{hyperref} 				% łącza
\usepackage{picture} 				% rysunki
\usepackage{upgreek}				% greckie symbole

\usepackage{listings}				% listingi
\usepackage{color}					% kolory


 % Polecenie usunięcia paginy na pustej stronie.
\newcommand{\clearemptydoublepage}{\newpage{\pagestyle{empty}\cleardoublepage}} 	


% Zmiana nazwy rozdziału z listingami, domyślnie jest Listings.
\renewcommand{\lstlistlistingname}{Spis listingów}
%%%%%%%%%%%%%%%%%%%%%%%%%%%%%%%%%%%%%%%%%%%%%
% koniec



% Dodatkowo %%%%%%%%%%%%%%%%%%%%%%%%%%%%%%%%%%%%%%%%%%%

% Definicje własnych otoczeń.
\theoremstyle{definition}
\newtheorem{dfn}{Definicja}[chapter]
\newtheorem{prz}[dfn]{Przykład}

\theoremstyle{remark}
\newtheorem{wsk}[dfn]{Wniosek}
% koniec


% Definiowanie własnego stylu listingu.
\lstset{
  frame             	= lines,
  basicstyle        	= \linespread{1}\ttfamily,%\small\footnotesize,
  columns           	= fullflexible,
  showstringspaces  = false,
  commentstyle      	= \color{gray}\upshape,
  numbers           	= left,
  numbersep         	= 5pt,
  stepnumber        	= 1,
  captionpos        	= t,
  numberstyle      	= \tiny,
}
%%%%%%%%%%%%%%%%%%%%%%%%%%%%%%%%%%%%%%%%%%%%%
% koniec	 



\begin{document}

%____________________________________________________________________________________________
%____________________________________________________________________________________________
% Strona tytułowa.
%____________________________________________________________________________________________

%%%%%%%%%%%%%%%%%%%%%%%%%%%%%%%%%%%%%%%%%%%%%
\makeatother 
% Tutaj wszystko należny wypełnić w nawiasach {}, nic więcej nie zmieniać!

\author{\textbf{Piotr Jasina}}        	% Podaj imię i nazwisko.
\title{Identyfikacja inteligentnych kontraktów w sieci Ethereum}      % Podaj tytuł polski.
\date{{\textbf{Lublin rok \the\year}}}	% Nic nie zmieniaj, rok wygeneruje automatycznie.
\def\engtitle{Ethereum smart contracts identification} % Podaj tytuł angielski.

\def\kierunek{Informatyka}		% Podaj kierunek studiów.
\def\specjalnosc{.....}			% Podaj specjalność, jeśli istnieje.
\def\nralbumu{279183}			% Podaj nr albumu.

\def\rodzaj{licencjacka}			% Podaj rodzaj pracy.
\def\zaklad{Zakładzie Cyberbezpieczeństwa}	% Podaj nazwę zakładu Twojego promotora.
\def\promotor{dr. Damiana Rusinka}	% Podaj promotora.

\makeatletter

%%%%%%%%%%%%%%%%%%%%%%%%%%%%%%%%%%%%%%%%%%%%%
% Poniżej w {titlepage} nic nie zmieniać.

\renewcommand{\maketitle}{
\begin{titlepage}
\begin{table}
\begin{tabular}{c@{\hspace{10mm}}|@{\hspace{3mm}}l}
\multicolumn{2}{@{\hspace{15mm}}l}{\vspace{-31mm}} \\
\multicolumn{2}{l}{\hspace{-20mm}\includegraphics[scale=0.4]{UMCS}} \\ 
\multicolumn{2}{@{\hspace{16mm}}l}{\vspace{7mm}} \\

\multicolumn{2}{@{\hspace{16mm}}l}{\Large UNIWERSYTET MARII CURIE-SKŁODOWSKIEJ} \\
\multicolumn{2}{@{\hspace{16mm}}l}{\vspace{-4mm}} \\
\multicolumn{2}{@{\hspace{16mm}}l}{\Large W LUBLINIE} \\
\multicolumn{2}{@{\hspace{16mm}}l}{\vspace{-4mm}} \\
\multicolumn{2}{@{\hspace{16mm}}l}{\Large Wydział Matematyki, Fizyki i Informatyki} \\
\multicolumn{2}{@{\hspace{16mm}}l}{\vspace{16mm}} \\

& \multicolumn{1}{m{13cm}}{Kierunek: \kierunek} \\
% Jeśli nie ma specjalności, to należy zakomentować poniższą linijkę!
% & \multicolumn{1}{m{13cm}}{Specjalność: \specjalnosc} \\
& \\
& \multicolumn{1}{m{13cm}}{\@author} \\
& \multicolumn{1}{m{13cm}}{nr albumu: \nralbumu} \\
& \\
& \\
& \multicolumn{1}{m{13cm}}{\Large\textbf{\@title}} \\
& \\
& \multicolumn{1}{m{13cm}}{\engtitle} \\
& \\
& \\
& \multicolumn{1}{m{13cm}}{Praca \rodzaj} \\	
& \vspace{-7mm} \\
& \multicolumn{1}{m{13cm}}{napisana w \zaklad} \\
& \vspace{-7mm} \\
& \multicolumn{1}{m{13cm}}{pod kierunkiem \promotor} \\
\multicolumn{2}{@{\hspace{20mm}}l}{\vspace{7mm}} \\
\multicolumn{2}{@{\hspace{20mm}}l}{\@date}
\end{tabular}
\end{table}
\end{titlepage}
}
% koniec 

% Tworzenie strony tytułowej.
\maketitle
% Wyczyszczenie z numeracji strony za stroną tytułową. 
\clearemptydoublepage


%____________________________________________________________________________________________
%____________________________________________________________________________________________
% Spis treści. 
%____________________________________________________________________________________________
\tableofcontents 		
% Wyczyszczenie strony z numeracji i paginy górnej, jeśli strona jest pusta. 	
\clearemptydoublepage			


%____________________________________________________________________________________________
%____________________________________________________________________________________________
% Wstęp pracy.
%____________________________________________________________________________________________
\chapter*{Wstęp}					% bez numeru rozdziału
\addcontentsline{toc}{chapter}{Wstęp} 	% bez numeru w spisie treści

...
% Wyczyszczenie strony z numeracji i paginy górnej, jeśli strona jest pusta. 
\clearemptydoublepage 


%____________________________________________________________________________________________
%____________________________________________________________________________________________
% Rozdział 1.
%____________________________________________________________________________________________
\chapter[Ethereum]{Ethereum}
% W nawiasach [] zapisujemy skróconą nazwę rozdziału, pojawiający się w paginie górnej oraz w spisie treści.
% W nawiasach {} zapisujemy pełną nazwę rozdziału.
% '\protect \\' służy do dzielenia długiej nazwy rozdziału.

\section{Historia}

Literatura: \cite{autor1, autor2}. TODO


\section{Opis platformy}

Literatura: \cite{autor1, autor2}. TODO

\section{Ethereum Virtual Machine}

Literatura: \cite{autor1, autor2}. TODO

\section{Inteligentne kontrakty}

Literatura: \cite{autor1, autor2}. TODO


% Wyczyszczenie strony z numeracji i paginy górnej, jeśli strona jest pusta. 
\clearemptydoublepage 


%____________________________________________________________________________________________
%____________________________________________________________________________________________
% Rozdział 2.
%____________________________________________________________________________________________
\chapter[Solidity]{Solidity}

\section{Sygnatura funkcji}

Literatura: \cite{autor1, autor2}. TODO


\section{Selektor funkcji}

Literatura: \cite{autor1, autor2}. TODO

\section{Generowanie akcesorów podczas kompilacji}

Literatura: \cite{autor1, autor2}. TODO


% Wyczyszczenie strony z numeracji i paginy górnej, jeśli strona jest pusta. 
\clearemptydoublepage 


%____________________________________________________________________________________________
%____________________________________________________________________________________________
% Rozdział 3.
%____________________________________________________________________________________________
\chapter[Projekt Aplikacji]{Projekt Aplikacji}
Celem mojej pracy licencjackiej było utworzenie aplikacji internetowej umożliwiającej identyfikacje  inteligentnych kontraktów wykorzystywanych w sieci Ethereum. Dzięki aplikacji użytkownik po wprowadzeniu na stronie kodu bajtowego kontraktu jest w stanie otrzymać najbardziej prawdopodobną implementacje kontraktu napisana w języku Solidity bazując na bazie danych aplikacji.

Poniżej zostało opisane działanie aplikacji wraz ze szczegółowym opisem funkcjonalności, architektury oraz wykorzystanych technologi.
\section{Funkcjonalność}
Po wejściu na stronę główną aplikacji użytkownik zobaczy w górnej części menu, w którym ma do wyboru: identyfikację inteligentnych kontraktu, wprowadzanie plików źródłowych kontraktów do aplikacji oraz dokumentacje API aplikacji. Na stronie głównej poniżej menu znajduje się opis aplikacji wraz z aktualna liczba kodów źródłowych znajdujących się w bazie danych aplikacji.

\subsection{Identyfikacja inteligentnych kontraktów}
Pierwsza opcja dostępną w menu jest identyfikacja inteligentnych kontraktów. Po naciśnięciu  przycisku na menu, użytkownik zostanie przekierowany na podstronę na której ma możliwość wprowadzenia kodu bajtowego w systemie szesnastkowym.

Podczas wprowadzania kodu istnieje możliwość wprowadzenia kodu z prefiksem "0x" oraz bez tego prefixu. Jeśli użytkownik poda kod z prefiksem to aplikacja podczas przetwarzania tego kodu zignoruje niepotrzebne znaki. Takie rozwiązanie zostało zastosowane w celu zapewnienia użytkownikowi większej wygody oraz komfortu w korzystaniu z aplikacji. Użytkownik nie będzie musiał zastanawiać się czy dodać prefiks, czy nie, ponieważ obie opcje są wspierane.

Pomyślne wprowadzone dane wykorzystywane do identyfikacji zatwierdzamy przyciskiem "Submit", a następnie po stronie serwerowej aplikacji rozpoczynany jest proces analizy wprowadzonego kodu bajtowego oraz wyszukiwane są najbardziej prawdopodobne implementacje. W rezultacie ---jak widzimy na rysunku TUTAJ BEDZIE ZDJECIE :D--- utrzymujemy listę wyszukanych implementacji posortowanych malejąco według współczynnika dopasowania. Jeśli klikniemy przyciskiem na jedną z wyświetlonych pozycji to w nowej karcie przeglądarki otworzy się podstrona z implementacją kontraktu wraz z podświetleniem składni języka Solidity.

\subsection{Wprowadzanie kodu źródłowego kontraktu do aplikacji}
Kolejna funkcjonalnością dostępną dla użytkownika jest możliwość dodania własnego kodu źródłowego kontraktu napisanego w języku Solidity. Opcja ta umożliwia użytkownikom wsparcie aktualnej bazy danych o kolejne kody źródłowe inteligentnych kontraktów, w wyniku takiego działania wszyscy pozostali użytkownicy maja większa szanse na precyzyjną identyfikacje kontraktu. Jak widać na rysunku --TUTAJ RYSUNEK--, ze względu na wygodę użytkowników korzystających z aplikacji, zostały utworzone dwie możliwości wprowadzania kodów źródłowych.

Pierwsza opcja umożliwia wprowadzenie lokalnego pliku zawierającego kod źródłowy z dysku komputera za przeglądarki internetowej.

Druga możliwością jest wklejenie kodu źródłowego bezpośrednio do pola tekstowego. Druga opcja została utworzona ponieważ, podczas korzystania z aplikacji użytkownik może bezpośrednio skopiować kod źródłowy, który jest w dowolnym innym źródle tekstowym i wkleić go bezpośrednio do mojej aplikacji bez konieczności tworzenia pliku tymczasowego.



\subsection{Interfejs programistyczny aplikacji}


\section{Architektura}

Literatura: \cite{autor1, autor2}. TODO

\subsection{Wyszukiwanie sygnatur funkcji w kodzie źródłowym}

Literatura: \cite{autor1, autor2}. TODO

\subsection{Wyszukiwanie selektorów funkcji w kodzie bajtowym}

Literatura: \cite{autor1, autor2}. TODO

\subsection{Szukanie implementacji na podstawie kodu bajtowego}

Literatura: \cite{autor1, autor2}. TODO

\section{Wykorzystane technologie}

Literatura: \cite{autor1, autor2}. TODO


% Wyczyszczenie strony z numeracji i paginy górnej, jeśli strona jest pusta. 
\clearemptydoublepage 




%____________________________________________________________________________________________
%____________________________________________________________________________________________
% Bibliografia.
%____________________________________________________________________________________________
\phantomsection
\begin{thebibliography}{99}
\phantomsection						% Służy do poprawnej nawigacji odnośnika w spisie treści.
\addcontentsline{toc}{chapter}{Bibliografia}		% Dodanie do spisu treści.

\bibitem{autor2} Bibliografia 1. \url{http://www.google.com}.
\bibitem{autor1} Bibliografia 2. \emph{Nazwa}. 

\end{thebibliography}

% Wyczyszczenie strony z numeracji i paginy górnej, jeśli strona jest pusta. 
\clearemptydoublepage					
% koniec

%____________________________________________________________________________________________
%____________________________________________________________________________________________
% Spis tabel.
%____________________________________________________________________________________________
\phantomsection						% Służy do poprawnej nawigacji odnośnika w spisie treści.	
\listoftables
\addcontentsline{toc}{chapter}{Spis tabel}		% Dodanie do spisu treści.
% Wyczyszczenie strony z numeracji i paginy górnej, jeśli strona jest pusta. 
\clearemptydoublepage
% koniec

%____________________________________________________________________________________________
%____________________________________________________________________________________________
% Spis rysunków.
%____________________________________________________________________________________________
\phantomsection						% Służy do poprawnej nawigacji odnośnika w spisie treści.
\listoffigures
\addcontentsline{toc}{chapter}{Spis rysunków}
% Wyczyszczenie strony z numeracji i paginy górnej, jeśli strona jest pusta. 
\clearemptydoublepage
% koniec

%____________________________________________________________________________________________%____________________________________________________________________________________________
% Spis listingów. 
%____________________________________________________________________________________________
\phantomsection						% Służy do poprawnej nawigacji odnośnika w spisie treści.
\lstlistoflistings
\addcontentsline{toc}{chapter}{Spis listingów} 
% Wyczyszczenie strony z numeracji i paginy górnej, jeśli strona jest pusta. 
\clearemptydoublepage
% koniec

\end{document}
